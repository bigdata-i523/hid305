\documentclass[sigconf]{acmart}

\usepackage{graphicx}
\usepackage{hyperref}
\usepackage{todonotes}

\usepackage{endfloat}
\renewcommand{\efloatseparator}{\mbox{}} % no new page between figures

\usepackage{booktabs} % For formal tables

\settopmatter{printacmref=false} % Removes citation information below abstract
\renewcommand\footnotetextcopyrightpermission[1]{} % removes footnote with conference information in first column
\pagestyle{plain} % removes running headers

\newcommand{\TODO}[1]{\todo[inline]{#1}}

\begin{document}
\title{Big Data applied to zoning and city planning}

\author{Andres Castro Benavides}
\orcid{1234-5678-9012}
\affiliation{%
  \institution{Indiana University}
  \streetaddress{107 S. Indiana Avenue}
  \city{Bloomington} 
  \state{Indiana} 
  \postcode{43017-6221}
}
\email{acastrob@iu.edu}

% The default list of authors is too long for headers}
\renewcommand{\shortauthors}{B. Trovato et al.}


\begin{abstract}
Developing city planning and zoning policies has a large impact on natural resources management and socioeconomic indicators. To make informed decisions, the local governments require to use important amounts of data to make informed decisions.

This paper explores the possibilities for zoning and city planning provided by the new big data technologies.

\end{abstract}

\keywords{Waste Management, Big Data, Local Government}



\maketitle

\section{Introduction}

Solid Waste Management (SWM) is a set of consistent and systematic regulations related to control generation, storage, collection, transportation, processing and land filling of wastes according to the best public health principles, economy, preservation of resources, aesthetics, other environmental requirements and what the public attends to ~\cite{akbarpour2016}

Managing solid waste is one of an essential services which often fails due to
rapid urbanization along with changes in the waste quantity and composition.
Quantity and composition vary from country to country making them difficult to
adopt for waste management system which may be successful at other places.
Quantity and composition of solid waste vary from place to place ~\cite{chandrappa2012} 


\section{ Zoning and City Planning}
%a) what is the problem



\section{ Variables Involved in Zoning and City Planning}
%b) why is big data involved ...  Use WU descrition of Big Data 


\section{Decision Making based on Facts}
%how can big data or analytics of big data help




\section{Existing Tools}
%what infrastructure/programs/systems exist for this

\subsection{GIS Analytics}


\subsection{Statistics}



\section{Conclusions}

Working on this



\appendix



\begin{acks}

 
\end{acks}

\bibliographystyle{ACM-Reference-Format}
\bibliography{report2} 

\end{document}
